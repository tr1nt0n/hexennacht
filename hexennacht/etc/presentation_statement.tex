\documentclass[11pt]{article}
\usepackage{fontspec}
\usepackage[utf8]{inputenc}
\setmainfont{Bodoni 72 Book}
\usepackage[paperwidth=11in,paperheight=17in,margin=1in,headheight=0.0in,footskip=0.5in,includehead,includefoot,portrait]{geometry}
\usepackage[absolute]{textpos}
\TPGrid[0.5in, 0.25in]{23}{24}
\parindent=0pt
\parskip=12pt
\usepackage{nopageno}
\usepackage{graphicx}
\graphicspath{ {./images/} }
\usepackage{amsmath}
\usepackage{tikz}
\newcommand*\circled[1]{\tikz[baseline=(char.base)]{
            \node[shape=circle,draw,inner sep=1pt] (char) {#1};}}

\begin{document}

\vspace*{8\baselineskip}

\begingroup
\begin{center}
\huge 03/14/22
\end{center}
\endgroup

\begingroup
\begin{center}
To reflect on past atrocity, whether that atrocity be a type of collective, or personal, is to exist in a position of partial temporal omnipresence and omniscience. At the time of reading, hearing, or thinking of past war, slavery, genocide, unjust execution, or any abhorrent violation of the innocent, one inhabits a kind of presence in that moment. I've read Elie Wiesel's harrowing masterpiece, \textit{Night}, three times so far. Each time, my pulse races, my eyes tear, and my heart breaks, at an intensity which increases with each reading. To review our knowledge of the ways in which our fellow human beings' bodies and souls have been massacred should not be a jading experience, but a visceral reminder of our empathy, compassion, and humanity. 
\end{center}
\endgroup

\begingroup
\begin{center}
To inhabit a kind of presence in a past moment invokes these viscera through our partial omniscience as future observers. It is its own horror to know of the brutality to come before it has transpired: to simultaneously premonish and remember, reorienting ourselves based on which portion of the timeline we are lead to exist in. Where are we, and who do we become, when we see an innocent young person dancing, one who has forgotten her suffering for a moment, as her body and mind join, and she sings and moves in an exercise of raw, intuitive engagement with the soul of her surroundings? And how do we witness her, when we know that in just a few moments, her door will be beaten and broken down, she will be apprehended, dragged, berated, defamed, and drowned, or burned, or hung by the neck until she is dead, for crimes she had no ability to commit, by a community who promised her a place, a role, and a belonging? Where are we, and who do we become, as our knowledge permits us to exist along all points of her dreadful story simultaneously?
\end{center}
\endgroup

\begingroup
\begin{center}
Who is the ensemble that stops the playing of the harp and the piano to shriek in a whirlwind of distorted images of what has come, and what will be? Who is that ensemble when their shrieking comes in its time, and runs its entire course, cycling in a delirium of repetition, and revealing its contents to be built entirely of fragments of the beings which played their role before itself? Who are the interrupted harpist and pianist, and what do they become, when they begin their playing together again and again, and start to realize that they will not finish their dance, without being swept away in the aforementioned whirlwind? Who is the bassoonist, and what do they become when they are surrounded by other sounds, always articulating themself in different manners of attack, different timbres, different starting points of the same line, but always the same melody, and the same rhythms? Who are the brass players when they hide their fluttering valves, flung arms, and heaving breaths beneath their surroundings, and who do they become the second, third, and fourth time, always beneath their sonic surroundings, but these surroundings and the duration of their presence subtly changing each time? Who do they become when their surroundings fall out from under them, and they are exposed, or revealed, or perhaps finally given their deserved attention, having permission to fully sing, and fully cry with one another? Who is the bass drum player when they beat the drum the first time, and who do they become the second, third, fourth, and fifth? And what do the flautist and oboist become when they proceed the brass, to close a moment rather than to begin one, and who is the flautist, playing without the oboe for the first time?
\end{center}
\endgroup

\begingroup
\begin{center}
Finally, who are we when we hear the final flourishes of the bassoon, the cello, and the gong? In the seconds before and perhaps during the applause when we, our bodies and minds, recall the events that have transpired before us? When we see the conductor beat the same metric pattern for the third time, this time being allowed to progress only one measure from the previous pattern before they are driven to silence by a new, ever-increasing tempo, are we different beings from the ones who saw them beat fragments of the same patterns at slower tempi just a moment before? Has the knowledge of our bodies brought us closer to omniscience, and has our ability to reflect brought us closer to omnipresence along moments of our past? When we are the ones whose bodies experience and whose souls ingest, can we also be the ones whose minds create?
\end{center}
\endgroup

\end{document}