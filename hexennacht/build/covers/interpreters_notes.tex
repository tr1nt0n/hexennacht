\documentclass[11pt]{article}
\usepackage{fontspec}
\usepackage[utf8]{inputenc}
\setmainfont{Bodoni 72 Book}
\usepackage[paperwidth=11in,paperheight=17in,margin=1in,headheight=0.0in,footskip=0.5in,includehead,includefoot,portrait]{geometry}
\usepackage[absolute]{textpos}
\TPGrid[0.5in, 0.25in]{23}{24}
\parindent=0pt
\parskip=12pt
\usepackage{nopageno}
\usepackage{graphicx}
\graphicspath{ {./images/} }
\usepackage{amsmath}
\usepackage{tikz}
\newcommand*\circled[1]{\tikz[baseline=(char.base)]{
            \node[shape=circle,draw,inner sep=1pt] (char) {#1};}}

\begin{document}

\vspace*{4\baselineskip}

\begingroup
\begin{center}
\huge NOTES FOR THE INTERPRETERS
\end{center}
\endgroup

\begingroup
\begin{center}
\leftskip0.5in
\textbf{General: 1.)} Dynamics in this score are effort dynamics, representing the force behind an action rather than the sounding dynamic. \textbf{2.)} Dashed arrows above the staff indicate a gradual transition from one technique to another. \textbf{3.)}  Stem tremoli are to be performed as quickly as possible, and do not represent a subdivision of a note. \\
\rightskip\leftskip
\end{center}
\endgroup

\begingroup
\begin{center}
\leftskip0.5in
\textbf{Winds and Woodwinds: 1.)} In extended passages where no breaths or rests are notated, interpreters are encouraged to break the line at their discretion. \textbf{2.)} Multiphonics are accompanied with fingering diagrams and approximate pitches in the staff. Not all notated pitches must sound. \\
\rightskip\leftskip
\end{center}
\endgroup

\begingroup
\begin{center}
\leftskip0.5in
\textbf{Flute: 1.)} Notes in the staff represent which keys to close rather than sounding pitch. \textbf{2.)} Head joint tilt is represented by degree articulations, wherein 0° indicates tilting the head joint parallel to the mouth, a la jet-whistle position, 45° indicates ordinario, and 90° indicates tilting the head joint perpendicular to the mouth, creating aeolian sound. \\ 
\rightskip\leftskip
\end{center}
\endgroup

\begingroup
\begin{center}
\leftskip0.5in
\textbf{Oboe: 1.)} Opened and closed articulations indicate covering or uncovering the bell with the hand. In the absence of these articulations, play ordinario. \\
\rightskip\leftskip
\end{center}
\endgroup

\begingroup
\begin{center}
\leftskip0.5in
\textbf{Trumpet: 1.)} Glissandi underneath trill spanners indicate to continue to trill while the embouchure glissandos to the next note. \\
\rightskip\leftskip
\end{center}
\endgroup

\begingroup
\begin{center}
\leftskip0.5in
\textbf{Horn and Trombone: 1.)} Notes in the staff represent embouchure rather than sounding pitch. \\
\rightskip\leftskip
\end{center}
\endgroup

\begingroup
\begin{center}
\leftskip0.5in
\textbf{Piano: 1.)} The piano should be prepared with thin chain and printer paper laid across all of the strings. \\
\rightskip\leftskip
\end{center}
\endgroup

\begingroup
\begin{center}
\leftskip0.5in
\textbf{Harp: 1.)} The harp is tuned to \textbf{C-sharp, D-natural, E-flat, F-sharp, G-natural, A-flat,} and \textbf{B-natural} for the entire piece. \textbf{2.)} The harpist should be equipped with a thick plastic card with which to scrape the wire-wrapped strings. \\
\rightskip\leftskip
\end{center}
\endgroup

\begingroup
\begin{center}
\leftskip0.5in
\textbf{Percussion: 1.) The first percussionist} plays a 5-octave marimba, and a small, medium, and large ratchet. \textbf{2.)} The marimba should be covered with towels across the length of the instrument, with the edges of the keys uncovered. When bowing the marimba, this damping should produce a subtle white noise. \textbf{3.)}  It is advised that the ratchets are mounted in front of the marimba. In the absence of stem tremoli, ratchets should be clicked once rather than turned continuously. \textbf{4.)} The first percussionist's implements are drumsticks and two bows. \textbf{5.) The second percussionist} plays a bass drum, a small, medium, and large gong, and a ride cymbal. \textbf{6.)} The bass drum head should be undamped. The head should be slackened such that when rubbing with the hand, a pitch can be produced with enough friction. When instructed to play the bass drum ``with hand,'' the interpreter should rub the drum in the aforementioned manner. \textbf{7.)} When bowing the gong, Col legno battuto (C.L.B.) indicates to strike with the wood of the bow, and Crine indicates to rub with the hair of the bow. \textbf{8.)} The second percussionist's implements are two bows, a triangle beater, two hard bass drum mallets, and a soft gong mallet. \\
\rightskip\leftskip
\end{center}
\endgroup

\begingroup
\begin{center}
\leftskip0.5in
\textbf{Strings: 1.)} Diamond-shaped noteheads indicate to touch the string on the notated pitch with harmonic pressure, regardless of if a harmonic sounds or not. Cross-shaped noteheads indicate to damp the strings, removing as much pitch as possible. \textbf{2.)} Flautando (Flaut.) indicates to bow quickly and with as light pressure as possible, Overpressure (O.P.) indicates to bow slowly and with as heavy bow pressure as possible, and Normale (Norm.) cancels Flautando and Overpressure. \textbf{3.)} Molto sul tasto (M.S.T.) indicates to bow on the strings above the fingerboard as close to the fingers as possible, Sul tasto (S.T.) indicates to bow above the edge of the fingerboard, Sul ponticello (S.P.) indicates to bow near the bridge, Molto sul ponticello (M.S.P.) indicates to bow with half of the bow hair directly on the bridge, and half of the bow hair directly on the strings, and Ordinario (Ord.) cancels all bow position instructions. \textbf{4.)} Col legno tratto (C.L.T.) indicates to rub the strings with the wood of the bow. Crine cancels Col legno tratto. \textbf{5.)} Justly tuned passages are notated using Helmholtz-Ellis accidentals with the deviation in cents from the closest equally tempered note printed above for use with an electric tuner. In the absence of tuners, approximations of these detunings are acceptable. \\
\rightskip\leftskip
\end{center}
\endgroup

\end{document}